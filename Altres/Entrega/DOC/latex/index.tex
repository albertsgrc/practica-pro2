En aquesta pràctica s'ha implementat un programa modular que ofereix un menú d'opcions per simular un experiment de laboratori amb {\bfseries organismes cel·lulars} que poden créixer, decréixer, morir i reproduir-\/se, a més de disposar de l'opció d'obtenir en tot moment un rànking amb informació de les reproduccions dels organismes.

Per implementar-\/lo s'ha dividit en {\bfseries 5 mòduls}\-: 
\begin{DoxyItemize}
\item El programa principal, el codi del qual està contingut en l'arxiu {\itshape \hyperlink{pro2_8cpp}{pro2.\-cpp}}. 
\item La classe \hyperlink{class_celula}{Celula} ({\itshape \hyperlink{_celula_8cpp}{Celula.\-cpp}}), per representar les cèl·lules dels organismes. 
\item La classe \hyperlink{class_organisme}{Organisme} ({\itshape \hyperlink{_organisme_8cpp}{Organisme.\-cpp}}), per representar els organismes i les seves operacions. 
\item La classe \hyperlink{class_ranking}{Ranking} ({\itshape \hyperlink{_ranking_8cpp}{Ranking.\-cpp}}), que representa el {\bfseries rànking de reproduccions} i les seves operacions. 
\item La classe \hyperlink{class_sistema}{Sistema} ({\itshape \hyperlink{_sistema_8cpp}{Sistema.\-cpp}}), que conté el {\bfseries conjunt d'organismes} i certa informació relativa a l'experiment. 
\end{DoxyItemize}

{\bfseries Comentaris\-:} 
\begin{DoxyItemize}
\item Sobre el diagrama modular, es pot notar que el programa principal no accedeix mai directament sobre la classe \hyperlink{class_organisme}{Organisme}. Això és degut a que podríem tenir una pèrdua notable d'eficiència si ho féssim, donat que els objectes de la classe \hyperlink{class_organisme}{Organisme} poden tenir un tamany (en memòria) important i en la majoria dels casos hauríem d'obtenir els organismes mitjançant la classe \hyperlink{class_sistema}{Sistema} (aquesta hauria de tenir funcions que retornéssin organismes), i això suposaria haver de fer una còpia cada cop que en volguéssim consultar.  
\item També es pot notar que hi ha operacions que en principi no haurien de modificar el \hyperlink{class_sistema}{Sistema} o els organismes, però aquests paràmetres es passen per {\bfseries referència no constant}. Això és degut a que si els passéssim per referència constant hauríem de fer còpies per consultar els arbres dels organismes, i això ens suposaria novament una pèrdua notable d'eficiència. En el seu defecte es passen per referència i per tal que no quedin modificats es replanten les arrels amb els subarbres en totes les funcions de consulta.  
\end{DoxyItemize}